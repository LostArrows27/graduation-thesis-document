\section{Bố cục trình bày}
Khóa luận được chia thành 5 chương, mỗi chương tập trung vào một nội dung chính trong quá trình xây dựng và phát triển ứng dụng Smart Gallery. Cụ thể như sau:

\textbf{Chương 2: Kiến thức cơ sở}. Trình bày những kiến thức nền tảng và công nghệ liên quan đến việc phát triển ứng dụng Smart Gallery. Các công nghệ về phía máy chủ và người dùng sẽ được giới thiệu. Đặc biệt, phần sau của chương tập trung vào các mô hình và thư viện AI đóng vai trò quan trọng trong ứng dụng như OpenCLIP, Gemini và Face Recognition.

\textbf{Chương 3: Phân tích và thiết kế hệ thống}. Trình bày phương pháp tiếp cận và thiết kế các ca sử dụng, cơ sở dữ liệu cho ứng dụng Smart Gallery. Chương này mô tả chi tiết về các đặc tả yêu cầu của hệ thống, các mô tả ca sử dụng và việc thiết kế cơ sở dữ liệu cho ứng dụng. 

\textbf{Chương 4: Triển khai và kiểm thử ứng dụng quản lý thư viện ảnh tích hợp AI tạo video}. Tập trung vào thực nghiệm và đánh giá hệ thống. Chương này trình bày về quy trình triển khai kiến trúc hệ thống, cách hệ thống tạo kịch bản cho video, các chức năng chính và giao diện của hệ thống, cũng như các quy trình kiểm thử cho ứng dụng. Phạm vi kiểm thử được trình bày sẽ bao gồm kiểm thử đơn vị, kiểm thử API và kiểm thử tương tác người dùng. 

\textbf{Chương 5: Kết luận}. Tóm tắt những kết quả đạt được của khóa luận, đánh giá những đóng góp của ứng dụng Smart Gallery trong việc giải quyết các vấn đề quản lý ảnh của người dùng. Những hạn chế hiện tại và hướng phát triển trong tương lai cũng được đề xuất để tiếp tục cải thiện hệ thống.