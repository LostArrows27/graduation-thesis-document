\section{Bố cục trình bày}

Phần còn lại của khóa luận được bố cục như sau: 

Chương 2 trình bày những kiến thức nền tảng và công nghệ liên quan đến việc phát triển ứng dụng Smart Gallery. Chương này giới thiệu về các công nghệ được sử dụng trong phát triển ứng dụng, bao gồm công nghệ phía người dùng (Front-end), công nghệ phía máy chủ (Back-end), công nghệ cơ sở dữ liệu và một số công nghệ bổ sung khác. Đặc biệt, phần sau của chương tập trung vào các mô hình và thư viện AI đóng vai trò quan trọng trong ứng dụng như OpenCLIP và Face Recognition, trong đó cơ sở lý thuyết và cách thức ứng dụng của các công nghệ này được phân tích chi tiết.

Chương 3 trình bày phương pháp tiếp cận và thiết kế giải pháp cho ứng dụng Smart Gallery. Chương này mô tả chi tiết về luồng làm việc của ứng dụng, kiến trúc hệ thống, cách thức triển khai các thuật toán phân loại ảnh thông minh, thuật toán nhận diện khuôn mặt và phương pháp tạo video slideshow từ bộ sưu tập ảnh. Những thách thức kỹ thuật và giải pháp đề xuất cũng được thảo luận.

Chương 4 tập trung vào thực nghiệm và đánh giá hệ thống. Chương này trình bày về quy trình thực nghiệm, dữ liệu sử dụng, các độ đo đánh giá và phân tích kết quả thực nghiệm. Hiệu suất của các thuật toán phân loại ảnh, nhận diện khuôn mặt và tạo video slideshow được đánh giá một cách chi tiết, cùng với các so sánh với giải pháp hiện có trên thị trường.

Cuối cùng, phần kết luận tóm tắt những kết quả đạt được của khóa luận, đánh giá những đóng góp của ứng dụng Smart Gallery trong việc giải quyết các vấn đề quản lý ảnh của người dùng. Những hạn chế hiện tại và hướng phát triển trong tương lai cũng được đề xuất để tiếp tục cải thiện hệ thống.