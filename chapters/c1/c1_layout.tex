\section{Bố cục trình bày}

Phần còn lại của khóa luận được trình bày theo bố cục như sau: 

Chương 2 trình bày những kiến thức nền tảng và công nghệ liên quan đến việc phát triển ứng dụng Smart Gallery. Chương này giới thiệu về các công nghệ được sử dụng trong phát triển ứng dụng, bao gồm công nghệ phía người dùng, phía máy chủ, cơ sở dữ liệu và một số công nghệ bổ sung khác. Đặc biệt, phần sau của chương tập trung vào các mô hình và thư viện AI đóng vai trò quan trọng trong ứng dụng như OpenCLIP và Face Recognition, trong đó cơ sở lý thuyết và cách thức ứng dụng của các công nghệ này được phân tích chi tiết.

Chương 3 trình bày phương pháp tiếp cận và thiết kế giải pháp cho ứng dụng Smart Gallery. Chương này mô tả chi tiết về các đặc tả yêu cầu của hệ thống, các mô tả ca sử dụng và việc thiết kế cơ sở dữ liệu cho ứng dụng. Các ca sử dụng sẽ được mô tả chi tiết kèm theo mô tả chi tiết, các biểu đồ luồng hoạt động và các biểu đồ tuần tự.

Chương 4 tập trung vào thực nghiệm và đánh giá hệ thống. Chương này trình bày về quy trình triển khai kiến trúc hệ thống, cách hệ thống tạo kịch bản cho video, các chức năng chính và giao diện của hệ thống, cũng như các quy trình kiểm thử cho ứng dụng. 

Cuối cùng, phần kết luận tóm tắt những kết quả đạt được của khóa luận, đánh giá những đóng góp của ứng dụng Smart Gallery trong việc giải quyết các vấn đề quản lý ảnh của người dùng. Những hạn chế hiện tại và hướng phát triển trong tương lai cũng được đề xuất để tiếp tục cải thiện hệ thống.