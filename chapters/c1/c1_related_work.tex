\section{Hiện trạng thị trường} 
Phần này đề cập đến việc phân tích hiện trạng thị trường và nêu bật tầm quan trọng của việc nghiên cứu công trình liên quan trong quá trình phát triển ứng dụng Smart Gallery.
\subsection{Thị trường trong nước và quốc tế}
Trên thị trường Việt Nam và quốc tế đã có không ít các ứng dụng giúp quản lý thư viện ảnh người dùng với các ưu điểm và hạn chế như sau:
\begin{itemize}
    \item \textbf{Google Photos}: Là một trong những ứng dụng quản lý ảnh phổ biến nhất hiện nay. Google Photos cho phép người dùng lưu trữ, chia sẻ và chỉnh sửa ảnh trực tuyến. Tuy nhiên, ứng dụng này yêu cầu người dùng phải có tài khoản Google và có giới hạn dung lượng lưu trữ miễn phí. Ngoài ra tính năng tạo video kỉ niệm của Google Photos còn khá đơn giản và không cho phép người dùng tùy biến nhiều.
    \item \textbf{Apple Photo}: là ứng dụng cung cấp sự tích hợp liền mạch trong hệ sinh thái Apple, tự động đồng bộ hóa ảnh trên iPhone, iPad và Mac thông qua iCloud. Nó cung cấp khả năng tổ chức thành album và tạo video ``kỷ niệm'' dựa trên các bức ảnh của người dùng, cùng với một bộ công cụ chỉnh sửa ảnh. Với việc giới thiệu Apple Intelligence trên các thiết bị mới hơn, nó hiện tự hào có khả năng tìm kiếm nâng cao bằng cách sử dụng ngôn ngữ tự nhiên. Một nhược điểm lớn là khả năng sử dụng hạn chế bên ngoài hệ sinh thái Apple, khiến nó kém lý tưởng hơn cho người dùng có thiết bị không phải của Apple. Giao diện web chỉ cung cấp các chức năng cơ bản. Hay tính năng tạo video kỉ niệm của Apple Photo cũng không cho phép người dùng tùy biến nhiều.
\end{itemize}

\subsection{Điểm mạnh so với các ứng dụng trên thị trường}
Đối với các ứng dụng quản lý ảnh hiện có trên thị trường, mặc dù cung cấp các tính năng cơ bản về lưu trữ và phân loại ảnh, nhưng vẫn còn nhiều hạn chế đáng kể. Google Photos bị giới hạn dung lượng lưu trữ miễn phí và thiếu tính năng tùy biến trong việc tạo video kỷ niệm. Apple Photos lại bị giới hạn trong hệ sinh thái riêng, gây khó khăn cho người dùng các thiết bị không phải Apple, đồng thời giao diện web chỉ cung cấp các chức năng cơ bản. Cả hai ứng dụng đều có tính năng tạo video kỷ niệm còn khá sơ sài, thiếu tính năng tùy biến và chỉ là video đơn giản với các hiệu ứng chuyển cảnh cơ bản.

Do đó, khóa luận này đề xuất Smart Gallery - một ứng dụng di động không chỉ kế thừa những ưu điểm mà còn khắc phục những hạn chế của các ứng dụng tiền nhiệm. Smart Gallery nổi bật với khả năng tự động phân loại và gắn nhãn ảnh theo nhiều tiêu chí như địa điểm, hoạt động và sự kiện dựa trên nội dung hình ảnh. Hệ thống tích hợp công nghệ nhận diện khuôn mặt, cho phép người dùng quản lý và đặt tên cho các nhân vật xuất hiện trong ảnh, từ đó tạo điều kiện thuận lợi cho việc tìm kiếm và phân loại.

Điểm đặc biệt của Smart Gallery so với các ứng dụng hiện có là tính năng tìm kiếm thông minh, cho phép người dùng tìm kiếm ảnh bằng văn bản hoặc giọng nói với nhiều tùy chọn lọc khác nhau như thời gian, album, nhân vật. Ngoài ra, Smart Gallery còn cung cấp khả năng tạo video slideshow từ bộ sưu tập ảnh với khả năng tùy chỉnh đa dạng như theme, nhạc nền, độ phân giải video và thời lượng.

Ngoài ra, Smart Gallery còn cung cấp tính năng quản lý địa điểm cho ảnh và hiển thị vị trí các bức ảnh trên bản đồ, giúp người dùng dễ dàng theo dõi và có cái nhìn tổng quan về những nơi đã đến và địa điểm chụp ảnh.