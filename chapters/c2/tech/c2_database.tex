\subsection{PostgreSQL}
PostgreSQL~\cite{postgresqldoc} là một hệ quản trị cơ sở dữ liệu quan hệ mã nguồn mở, được phát triển từ năm 1986 tại Đại học California, Berkeley. Hệ thống này nổi bật với khả năng tuân thủ chặt chẽ các tiêu chuẩn SQL và hỗ trợ đầy đủ các giao dịch ACID (Atomicity, Consistency, Isolation, Durability). PostgreSQL cung cấp nhiều tính năng nâng cao như hỗ trợ các kiểu dữ liệu phức tạp, khả năng tạo kiểu dữ liệu tùy chỉnh, hàm và thủ tục lưu trữ trong nhiều ngôn ngữ lập trình. Điểm mạnh của PostgreSQL còn thể hiện ở khả năng mở rộng cao thông qua hệ thống extension phong phú, hỗ trợ truy vấn không gian địa lý qua PostGIS và khả năng xử lý dữ liệu phi cấu trúc với JSON/JSONB. 

\subsection{Supabase}
Supabase~\cite{supabasedoc} là một nền tảng backend-as-a-service mã nguồn mở, được phát triển như một giải pháp thay thế cho Firebase từ năm 2020. Nền tảng này cung cấp các dịch vụ toàn diện bao gồm cơ sở dữ liệu PostgreSQL được quản lý đầy đủ, hệ thống xác thực người dùng, lưu trữ tệp và API tự động. Supabase nổi bật với khả năng cung cấp API RESTful và GraphQL tự động dựa trên cấu trúc bảng PostgreSQL, kết hợp với SDK đa nền tảng giúp đơn giản hóa việc tích hợp. Nền tảng còn cung cấp các tính năng real-time thông qua Realtime Server dựa trên công nghệ Phoenix, cho phép theo dõi các thay đổi cơ sở dữ liệu trong thời gian thực. Supabase hỗ trợ Row Level Security (RLS) để quản lý quyền truy cập dữ liệu chi tiết đến từng hàng, cùng với hệ thống edge functions để thực thi mã serverless. 