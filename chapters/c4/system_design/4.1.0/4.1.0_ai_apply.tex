\subsection{Ứng dụng các mô hình và thư viện AI trong hệ thống}

\subsubsection{Mô hình Open-CLIP}
Trong hệ thống, OpenCLIP được triển khai để thực hiện hai chức năng chính: gán nhãn tự động cho hình ảnh và tìm kiếm hình ảnh dựa trên văn bản. Cụ thể hơn, Smart Gallery sử dụng một mô hình OpenCLIP được huấn luyện sẵn, kết hợp với các tập nhãn được định nghĩa trước để phân loại hình ảnh theo các nhãn có độ tương đồng cao nhất.

\textbf{Gán nhãn tự động cho hình ảnh:} Quá trình gán nhãn tự động được thực hiện theo các bước sau:

\begin{enumerate}
    \item \textbf{Chuẩn bị tập nhãn:} Smart Gallery sử dụng ba tập nhãn chính: địa điểm (location), hoạt động (activities), sự kiện (events), và một tập nhãn đặc biệt để phân loại ảnh có liên quan/không liên quan (relate/unrelate). Các tập nhãn này được mã hóa (encode) bằng bộ mã hóa văn bản của OpenCLIP và được lưu dưới dạng file .pt để sử dụng lại, tránh việc phải thực hiện mã hóa lặp lại.
    
    \item \textbf{Lọc ảnh không liên quan:} Khi một ảnh được tải lên, bước đầu tiên là sử dụng tập nhãn đặc biệt để loại bỏ các ảnh không liên quan như giấy tờ, hình ảnh hoạt hình hoặc ảnh chụp màn hình. Điều này giúp giảm thiểu việc gán nhãn không chính xác và tập trung vào các bức ảnh có giá trị cho người dùng.
    
    \item \textbf{Tính toán độ tương đồng và gán nhãn:} Đối với các ảnh đã được lọc, hệ thống sẽ mã hóa ảnh thành vector đặc trưng (image features) và tính toán độ tương đồng với các vector đặc trưng của từng nhãn trong các tập nhãn. Quá trình này bao gồm việc chuẩn hóa các vector đặc trưng và sử dụng phép nhân ma trận để tính toán xác suất tương đồng. Sau đó, hệ thống chọn ra top 2 nhãn có độ tương đồng cao nhất cho mỗi tập nhãn locations, activities, và events.
    
    \item \textbf{Lưu trữ kết quả:} Sau khi hoàn tất việc gán nhãn, cả vector đặc trưng của ảnh (image features) và các nhãn được gán sẽ được lưu vào cơ sở dữ liệu. Việc lưu trữ vector đặc trưng giúp tối ưu hóa quá trình tìm kiếm sau này.
\end{enumerate}

\textbf{Tìm kiếm hình ảnh dựa trên văn bản:} Khi người dùng nhập một câu truy vấn tìm kiếm, hệ thống sẽ thực hiện các bước sau:

\begin{enumerate}
    \item Mã hóa câu truy vấn thành vector đặc trưng văn bản (text features) sử dụng bộ mã hóa văn bản của OpenCLIP.
    
    \item Tính toán độ tương đồng cosine giữa vector đặc trưng văn bản và các vector đặc trưng hình ảnh đã được lưu trữ trong cơ sở dữ liệu.
    
    \item Trả về các kết quả có độ tương đồng vượt qua một ngưỡng nhất định, đảm bảo kết quả tìm kiếm có độ chính xác cao.
\end{enumerate}

Cách tiếp cận này cho phép Smart Gallery cung cấp khả năng tìm kiếm hình ảnh thông minh. Người dùng có thể tìm kiếm hình ảnh bằng các mô tả tự nhiên như ``bữa tiệc sinh nhật'', ``chuyến đi biển'' hoặc ``buổi picnic ở công viên'', và hệ thống sẽ trả về các hình ảnh phù hợp với mô tả đó.

\subsubsection{Thư viện Face Recognition}

Trong hệ thống, thư viện Face Recognition được ứng dụng để phát hiện và phân loại khuôn mặt xuất hiện trong thư viện ảnh của người dùng, qua hai chức năng chính:

\textbf{Nhận diện khuôn mặt trong ảnh:} Quá trình nhận diện khuôn mặt được thực hiện theo các bước sau:
\begin{enumerate}
    \item \textbf{Tiền xử lý ảnh:} Với những ảnh có kích thước lớn, hệ thống thực hiện giảm kích thước để tối ưu hóa hiệu suất xử lý mà vẫn đảm bảo độ chính xác trong việc nhận diện.
    
    \item \textbf{Trích xuất đặc trưng khuôn mặt:} Sử dụng thư viện face\_recognition để phát hiện vị trí các khuôn mặt trong ảnh và tính toán vector đặc trưng (face\_embedding) cho mỗi khuôn mặt.
    
    \item \textbf{Lưu trữ thông tin:} Hệ thống lưu thông tin về vị trí (location) của khuôn mặt trong ảnh cùng với vector đặc trưng tương ứng vào cơ sở dữ liệu để sử dụng cho việc phân nhóm và nhận diện sau này.
\end{enumerate}

\textbf{Phân nhóm khuôn mặt tương tự: } Để người dùng có thể dễ dàng quản lý và phân loại khuôn mặt xuất hiện trong thư viện ảnh, Smart Gallery thực hiện phân nhóm các khuôn mặt tương tự như sau:

\begin{enumerate}
    \item \textbf{Áp dụng thuật toán phân cụm:} Hệ thống sử dụng thuật toán DBSCAN (Density-Based Spatial Clustering of Applications with Noise)~\cite{dbscan} để phân cụm các vector đặc trưng khuôn mặt. DBSCAN được lựa chọn vì khả năng phát hiện các cụm với hình dạng tùy ý và không yêu cầu số lượng cụm trước.
    
    \item \textbf{Tính toán vector đại diện:} Sau khi phân cụm, hệ thống tính toán centroid (vector trung bình) cho mỗi nhóm khuôn mặt để làm đại diện cho nhóm đó.
    
    \item \textbf{Lưu trữ và đặt tên mặc định:} Các centroid được lưu vào cơ sở dữ liệu cùng với tên mặc định cho mỗi nhóm khuôn mặt.
    
    \item \textbf{Cập nhật thông tin nhóm:} Khi có thêm khuôn mặt mới được phát hiện, hệ thống thực hiện phân cụm lại và tính toán centroid mới với tập các khuôn mặt mới và khuôn mặt đã được phát hiện. Sau đó, hệ thống tính độ tương đồng cosine~\cite{cosine} giữa các centroid mới và các centroid đã lưu trước đó. Nếu độ tương đồng vượt quá ngưỡng định trước, hệ thống giữ nguyên centroid cũ và cập nhật thông tin nhóm cho các khuôn mặt mới. Cách tiếp cận này đảm bảo sự nhất quán trong việc phân loại khuôn mặt theo thời gian và giúp các nhóm khuôn mặt được cập nhật liên tục khi có thêm dữ liệu mới. 
\end{enumerate}

Với cách tiếp cận này, Smart Gallery có thể tự động nhận diện và tổ chức các khuôn mặt xuất hiện trong thư viện ảnh của người dùng, cho phép người dùng dễ dàng xem, tìm kiếm và phân loại ảnh theo các cá nhân xuất hiện trong đó. Người dùng cũng có thể đặt tên cho các nhóm khuôn mặt và tìm kiếm các ảnh bằng tên người đã đặt.